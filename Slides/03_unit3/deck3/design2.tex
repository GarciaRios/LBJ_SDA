% -*- TeX-engine: xetex; eval: (auto-fill-mode 0); eval: (visual-line-mode 1); -*-
% Compile with XeLaTeX

%%%%%%%%%%%%%%%%%%%%%%%
% Option 1: Slides: (comment for handouts)   %
%%%%%%%%%%%%%%%%%%%%%%%

\documentclass[slidestop,compress,mathserif,12pt,t,professionalfonts,xcolor=table]{beamer}

% solution stuff
\newcommand{\solnMult}[1]{
\only<1>{#1}
\only<2->{\red{\textbf{#1}}}
}
\newcommand{\soln}[1]{\textit{#1}}

%%%%%%%%%%%%%%%%%%%%%%%%%%%%%%%
% Option 2: Handouts, without solutions (post before class)    %
%%%%%%%%%%%%%%%%%%%%%%%%%%%%%%%

% \documentclass[11pt,containsverbatim,handout,xcolor=xelatex,dvipsnames,table]{beamer}

% % handout layout
% \usepackage{pgfpages}
% \pgfpagesuselayout{4 on 1}[letterpaper,landscape,border shrink=5mm]

% % solution stuff
% \newcommand{\solnMult}[1]{#1}
% \newcommand{\soln}[1]{}

%%%%%%%%%%%%%%%%%%%%%%%%%%%%%%%%%%%%
% Option 3: Handouts, with solutions (may post after class if need be)    %
%%%%%%%%%%%%%%%%%%%%%%%%%%%%%%%%%%%%

% \documentclass[11pt,containsverbatim,handout,xcolor=xelatex,dvipsnames,table]{beamer}

% % handout layout
% \usepackage{pgfpages}
% \pgfpagesuselayout{4 on 1}[letterpaper,landscape,border shrink=5mm]

% % solution stuff
% \newcommand{\solnMult}[1]{\red{\textbf{#1}}}
% \newcommand{\soln}[1]{\textit{#1}}

%%%%%%%%%%
% Load style file, defaults  %
%%%%%%%%%%

\input{../../lec_style.tex}
% Course Name
\newcommand{\CourseName}{LBJ - SDA - Spring 2024}
\newcommand{\InstituteName}{University of Texas}

% Personal Info
\newcommand{\FirstName}{Sergio}
\newcommand{\LastName}{Garcia-Rios}

% Electronic Info
\newcommand{\PersonalSite}{https://garciarios.github.io}
\newcommand{\CourseSite}{http://garciarios.github.io/govt_3990/}
\newcommand{\Email}{garcia.rios@cornell.edu}

% Exam Dates
\newcommand{\ExamADate}{Feb 24, Wed}
\newcommand{\ExamBDate}{Mar 30, Wed}
\newcommand{\FinalDate}{May 5, Thu - 7-10pm}
% ALT ALT
% \input{../../definitions_custom.tex}

%%%%%%%%%%%
% Cover slide info    %
%%%%%%%%%%%

\title{Survey Design Part 2}
\subtitle{Concepts and Questions}
\author{\CourseName}
\date{}
\institute{\InstituteName}


%%%%%%%%%%%%%%%%%%%%%%%%%
% Begin document and set Helvetica Neue font   %
%%%%%%%%%%%%%%%%%%%%%%%%%

\begin{document}
%\fontspec[Ligatures=TeX]{Helvetica Neue Light}

%%%%%%%%%%%%%%%%%%%%%%%%%%%%%%%%%%%

% Title Page

\begin{frame}[plain]

\titlepage

\vfill

{\scriptsize \webLink{\PersonalSite}{Dr. \LastName{}} \hfill Slides posted at  \webURL{\CourseSite}}

\addtocounter{framenumber}{-1} 

\end{frame}



%%%%%%%%%%%%%%%%%%%%%%%%%%%%%%%%%%%%


\begin{frame}{
Further refinement of concepts}

\begin{itemize}
\item Concepts contain specific elements within them that relate to the broader idea of the concept.
\item Concepts are composed of dimensions, sub-dimensions, and further sub-dimensions. 
\item Dimensions and sub-dimensions vary in relevance to the needs of the research. 
\item Identifying the most relevant dimensions and sub-dimensions of a concept is another step in clarifying concepts for measurement.
\end{itemize}
\end{frame}

\begin{frame}{Dimensions}

\begin{itemize}

\item Different aspects of a common core experience 

\item Clear indication of what the concept entails – how broad it is

\item Specify the most relevant aspects of a concept to the research question. 

\item What are key dimensions of: 
\begin{itemize}


\item Poverty?
\item Homelessness?
\item Middle Class?
\item Health? 
\item Education?
\item Employment?

\end{itemize}
\end{itemize}

\end{frame}

\begin{frame}{Sub-dimensions}
A sub-dimension is a further refinement of a dimension of a concept 

Similar in nature to dimensions, but further reflect layers of complexity to do with a concept 
Identifying SD’s allow greater focus and greater clarity for operationalization

Clearer way to develop valid and reliable indicators
\begin{itemize}
 
\item Health ->  mental health  ->  personality disorders

\end{itemize}

What are the sub-dimensions of:

\begin{itemize}
\item Poverty? 
\item Homelessness?
\item Middle class?

\end{itemize}
\end{frame}

\begin{frame}{Deciding on dimensions and sub-dimensions}


Context – place, institutional, social, political, economic
Relevance  - Which dimensions and sub-dimensions are relevant?

Necessity – Which are necessary and which are not?

Research question specific – Are they intuitive or clearly embedded in your research question?

Previous research – What has been done before? Do you need to innovate or expand on a concept and include dimensions and sub-dimensions?
\end{frame}


\begin{frame}{
Indicators}
The actual questions that populate a survey 
Questions that are measures of the concepts (dimensions and sub-dimensions) embedded in research questions and theories


Indicators need to be valid and reliable 

\begin{itemize}
\item Validity – that the question measures exactly what it is supposed to measure
\item Reliability – as a measure it produces similar results under similar conditions 
\end{itemize}

\end{frame}


\begin{frame}{
Validity}

Three forms of validity for indicators
Content validity – specific content, especially in dimensions and sub-dimensions 


Criterion/expert validity – test against old measures, consult with relevant groups when developing an indicator 


Construct validity – measures based on theories – is the measure sound/problematic or is the theory

\end{frame}

\begin{frame}{
Reliability}

A number of issues impact reliability
Indicator construction – consistency, clear wording, and coding


Recall – issues with memory in answering questions


Testing and retesting – repeating questions to a sample should garner the same responses  

\end{frame}

\begin{frame}{
Indicators}
Considerations: 

\begin{itemize}

\item How many indicators – need vs. want 
\item Design of indictors – wording and comprehension
\item Arrangement of indicators in the survey – strategically placed so respondents can follow, continue, and complete the survey 

\end{itemize}

\end{frame}

%%%%%%%%%%%%%%%%%%%%%%%%%%%%%%%%%%%

\end{document}